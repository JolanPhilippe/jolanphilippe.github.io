\begin{tikzpicture}

    % \node[] (commit) {Commit Code};
    \onslide<1>{}
    \onslide<2->{
    \node[] (packaging) {
    \begin{tabular}{c}
    \textbf{Management}\\
    \textbf{d'application}
    \end{tabular}
    };
    \node[below=0cm of packaging] (packagingL) {
    \begin{tabular}{c}
    \small Customiser, configurer\\
    \small tester l'application\\
    \small et la conteneuriser\\
    \small 
    \end{tabular}
    };
    }
    % ----------
\onslide<3->{
    \node[right=0.8cm of packaging] (provisioning) {
    \begin{tabular}{c}
    \textbf{Provisionnement} \\ 
    \textbf{d'infrastructure}
    \end{tabular}
    };
    \node[below=0cm of provisioning] (provisioningL) {
    \begin{tabular}{c}
    \small Demander ressources\\
    \small physiques ou virtuelles;\\
    \small configurer le réseau;\\
    \small et règles sécurité
    \end{tabular}
    };
    }
    
    % ----------
\onslide<4->{

    \node[right=0.8cm of provisioning] (configmanagement) {
    \begin{tabular}{c}
    \textbf{Installation et}\\
    \textbf{Configuration}
    \end{tabular}
    };
    \node[below=0cm of configmanagement] (configmanagementL) {
    \begin{tabular}{c}
    \small Installer les services\\
    \small (app + deps)\\
    \small configurer les services;\\
    \small et les intégrer
    \end{tabular}
    };
    }

    % -------
\onslide<5->{

    \node[right=0.8cm of configmanagement] (orchestration) {
    \begin{tabular}{c}
    \textbf{Orchestration}\\
    \textbf{de cycle de vie}
    \end{tabular}
    };
    \node[below=0cm of orchestration] (orchestrationL) {
    \begin{tabular}{c}
    \small Upgrades auto;\\
    \small Backup et recovery;\\
    \small Surveillance;\\
    \small Passage à l'échelle
    \end{tabular}
    };
    }
    
    
\end{tikzpicture}